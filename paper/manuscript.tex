% Options for packages loaded elsewhere
% Options for packages loaded elsewhere
\PassOptionsToPackage{unicode}{hyperref}
\PassOptionsToPackage{hyphens}{url}
\PassOptionsToPackage{dvipsnames,svgnames,x11names}{xcolor}
%
\documentclass[
  english,
  12pt,
  a4paper,
]{article}
\usepackage{xcolor}
\usepackage[top=1.5in,bottom=1.5in,left=1in,right=1in]{geometry}
\usepackage{amsmath,amssymb}
\setcounter{secnumdepth}{5}
\usepackage{iftex}
\ifPDFTeX
  \usepackage[T1]{fontenc}
  \usepackage[utf8]{inputenc}
  \usepackage{textcomp} % provide euro and other symbols
\else % if luatex or xetex
  \usepackage{unicode-math} % this also loads fontspec
  \defaultfontfeatures{Scale=MatchLowercase}
  \defaultfontfeatures[\rmfamily]{Ligatures=TeX,Scale=1}
\fi
\usepackage{lmodern}
\ifPDFTeX\else
  % xetex/luatex font selection
\fi
% Use upquote if available, for straight quotes in verbatim environments
\IfFileExists{upquote.sty}{\usepackage{upquote}}{}
\IfFileExists{microtype.sty}{% use microtype if available
  \usepackage[]{microtype}
  \UseMicrotypeSet[protrusion]{basicmath} % disable protrusion for tt fonts
}{}
\usepackage{setspace}
\makeatletter
\@ifundefined{KOMAClassName}{% if non-KOMA class
  \IfFileExists{parskip.sty}{%
    \usepackage{parskip}
  }{% else
    \setlength{\parindent}{0pt}
    \setlength{\parskip}{6pt plus 2pt minus 1pt}}
}{% if KOMA class
  \KOMAoptions{parskip=half}}
\makeatother
% Make \paragraph and \subparagraph free-standing
\makeatletter
\ifx\paragraph\undefined\else
  \let\oldparagraph\paragraph
  \renewcommand{\paragraph}{
    \@ifstar
      \xxxParagraphStar
      \xxxParagraphNoStar
  }
  \newcommand{\xxxParagraphStar}[1]{\oldparagraph*{#1}\mbox{}}
  \newcommand{\xxxParagraphNoStar}[1]{\oldparagraph{#1}\mbox{}}
\fi
\ifx\subparagraph\undefined\else
  \let\oldsubparagraph\subparagraph
  \renewcommand{\subparagraph}{
    \@ifstar
      \xxxSubParagraphStar
      \xxxSubParagraphNoStar
  }
  \newcommand{\xxxSubParagraphStar}[1]{\oldsubparagraph*{#1}\mbox{}}
  \newcommand{\xxxSubParagraphNoStar}[1]{\oldsubparagraph{#1}\mbox{}}
\fi
\makeatother


\usepackage{longtable,booktabs,array}
\usepackage{calc} % for calculating minipage widths
% Correct order of tables after \paragraph or \subparagraph
\usepackage{etoolbox}
\makeatletter
\patchcmd\longtable{\par}{\if@noskipsec\mbox{}\fi\par}{}{}
\makeatother
% Allow footnotes in longtable head/foot
\IfFileExists{footnotehyper.sty}{\usepackage{footnotehyper}}{\usepackage{footnote}}
\makesavenoteenv{longtable}
\usepackage{graphicx}
\makeatletter
\newsavebox\pandoc@box
\newcommand*\pandocbounded[1]{% scales image to fit in text height/width
  \sbox\pandoc@box{#1}%
  \Gscale@div\@tempa{\textheight}{\dimexpr\ht\pandoc@box+\dp\pandoc@box\relax}%
  \Gscale@div\@tempb{\linewidth}{\wd\pandoc@box}%
  \ifdim\@tempb\p@<\@tempa\p@\let\@tempa\@tempb\fi% select the smaller of both
  \ifdim\@tempa\p@<\p@\scalebox{\@tempa}{\usebox\pandoc@box}%
  \else\usebox{\pandoc@box}%
  \fi%
}
% Set default figure placement to htbp
\def\fps@figure{htbp}
\makeatother



\ifLuaTeX
\usepackage[bidi=basic]{babel}
\else
\usepackage[bidi=default]{babel}
\fi
% get rid of language-specific shorthands (see #6817):
\let\LanguageShortHands\languageshorthands
\def\languageshorthands#1{}
\ifLuaTeX
  \usepackage[english]{selnolig} % disable illegal ligatures
\fi


\setlength{\emergencystretch}{3em} % prevent overfull lines

\providecommand{\tightlist}{%
  \setlength{\itemsep}{0pt}\setlength{\parskip}{0pt}}



 
\usepackage[]{natbib}
\bibliographystyle{apalike}


\usepackage{amsmath, amsthm, amssymb}
\usepackage{booktabs}
\usepackage{tikz}
\usepackage{xcolor}
\usepackage{txfonts}

\usepackage{bbm}
\usepackage[colorlinks = true, urlcolor = black, linkcolor = black, citecolor = black]{hyperref}
\usepackage{caption, longtable, colortbl, array}
\captionsetup{labelfont=bf, textfont=normal, labelsep=colon}
\usepackage{etoolbox}
\makeatletter
\patchcmd{\maketitle}{\thispagestyle{plain}}{\thispagestyle{empty}}{}{}
\makeatother
\usepackage[bottom]{footmisc}
\interfootnotelinepenalty=10000
\usepackage[section]{placeins}
\makeatletter
\@ifpackageloaded{caption}{}{\usepackage{caption}}
\AtBeginDocument{%
\ifdefined\contentsname
  \renewcommand*\contentsname{Table of contents}
\else
  \newcommand\contentsname{Table of contents}
\fi
\ifdefined\listfigurename
  \renewcommand*\listfigurename{List of Figures}
\else
  \newcommand\listfigurename{List of Figures}
\fi
\ifdefined\listtablename
  \renewcommand*\listtablename{List of Tables}
\else
  \newcommand\listtablename{List of Tables}
\fi
\ifdefined\figurename
  \renewcommand*\figurename{Figure}
\else
  \newcommand\figurename{Figure}
\fi
\ifdefined\tablename
  \renewcommand*\tablename{Table}
\else
  \newcommand\tablename{Table}
\fi
}
\@ifpackageloaded{float}{}{\usepackage{float}}
\floatstyle{ruled}
\@ifundefined{c@chapter}{\newfloat{codelisting}{h}{lop}}{\newfloat{codelisting}{h}{lop}[chapter]}
\floatname{codelisting}{Listing}
\newcommand*\listoflistings{\listof{codelisting}{List of Listings}}
\makeatother
\makeatletter
\makeatother
\makeatletter
\@ifpackageloaded{caption}{}{\usepackage{caption}}
\@ifpackageloaded{subcaption}{}{\usepackage{subcaption}}
\makeatother
\usepackage{bookmark}
\IfFileExists{xurl.sty}{\usepackage{xurl}}{} % add URL line breaks if available
\urlstyle{same}
\hypersetup{
  pdftitle={Market responses to a VIX shock},
  pdfauthor={Luis Gruber, Gregor Kastner, Stefan Voigt, and Patrick Weiss},
  pdflang={en},
  pdfkeywords={Liquidity, variance risk premium, intraday dynamics, big
data, exchange-traded funds.},
  colorlinks=true,
  linkcolor={blue},
  filecolor={Maroon},
  citecolor={Blue},
  urlcolor={Blue},
  pdfcreator={LaTeX via pandoc}}


\title{Market responses to a VIX shock\thanks{Nikolaus Hautsch,
University of Vienna, Department of Statistics and Operations Research,
Research Platform Data Science @ Uni Vienna, Vienna Graduate School of
Finance (VGSF) and Center for Financial Studies (CFS),
\href{mailto:nikolaus.hautsch@univie.ac.at}{nikolaus.hautsch@univie.ac.at}.
Albert J. Menkveld, Vrije Universiteit Amsterdam,
\href{mailto:albertjmenkveld@gmail.com}{albertjmenkveld@gmail.com}.
Stefan Voigt, University of Copenhagen and Danish Finance Institute,
\href{mailto:stefan.voigt@econ.ku.dk}{stefan.voigt@econ.ku.dk}. Stefan
Voigt gratefully acknowledges support from the Danish Finance Institute
(DFI). We thank participants at the Annual Meeting of the Swiss Society
for Financial Market Research, the Fourteenth Annual SoFiE Conference,
5th Future of Financial Information Conference, the 15th International
Conference on Computational and Financial Econometrics (CFE), Financial
Intermediation Research Society (FIRS) 2022, the University of Toronto,
Lancaster University Management School, VU Amsterdam, Universidad EAFIT,
the Danish Finance Institute, the QFFE 2022, the Advances in Financial
Econometrics Conference at CBS and the Finance \& Financial Econometrics
Group at CREST. Menkveld is grateful to the Dutch Research Council (NWO)
for a Vici grant.}}
\author{Luis Gruber, Gregor Kastner, Stefan Voigt, and Patrick
Weiss\thanks{
  Luis Gruber is at University of Klagenfurt, e-mail: \href{mailto:luis.gruber@aau.at}{luis.gruber@aau.at}.
  Gregor Kaster is at University of Klagenfurt, e-mail: \href{mailto:gregor.kastner@aau.at}{gregor.kastner@aau.at}.
  Stefan Voigt is at University of Copenhagen and the Danish Finance Institute, e-mail: \href{mailto:stefan.voigt@econ.ku.dk}{stefan.voigt@econ.ku.dk}.
  Patrick Weiss is at Reykjavik University, e-mail: \href{mailto:patrickw@ru.is}{patrickw@ru.is}.
}}
\date{January 5, 2026}
\begin{document}
\maketitle
\begin{abstract}
Implied variance (VIX) shocks can be caused by either a perceived higher
expected future realized variance or an increase in the variance risk
premium. We analyze twenty billion NASDAQ order book messages for equity
and government-bond exchange-traded funds to delineate how the market
responds to shocks in either of these two components. We find that
investors actively sell equities and buy government bonds on largely
unchanged liquidity in response to an elevated price of variance
risk,e.g.~due to heightened crash-risk probabilities. The response to an
expected realized variance shock is active \emph{buying} of equities on
worse liquidity. We speculate that this pattern might be caused by an
information shock.
\end{abstract}

{\small\noindent\textbf{Keywords:} Liquidity, variance risk premium, intraday dynamics, big data, exchange-traded funds.\\
 \noindent\textbf{JEL Classification:} G11, G12, G20}
\clearpage
\setcounter{page}{1}


\setstretch{1.5}
\section{Introduction}\label{introduction}

Investors, regulators, and the media carefully monitor the \emph{VIX}
index, often referred to as a fear gauge \citep{carr17}. The VIX is an
estimate of the option-implied variance of the S\&P 500 portfolio
return. Since it is computed based on option prices it reflects the
(estimated) market's risk-neutral expectation of the S\&P 500's future
realized variance. As such, the \emph{VIX}aggregates information about
the market participant's expectation \emph{and} valuation of future
risks. The first component, the expected realized variance (\emph{ERV}),
predicts future return variation. The second component stems from the
sensitivity of risk-averse investors to extreme losses. Since risk
aversion elevates the risk-neutral probability of crash states,\\
the risk-neutral expectation of the realized variance typically differs
from the expected realized variance. The resulting wedge is commonly
referred to as the variance risk premium (\emph{VRP}).

How do markets respond to an unexpected movement in the \emph{VIX}? We
argue that the nature of such a shock triggers fundamentally different
responses: A shock in the \emph{VRP} indicates higher perceived
crash-risk probabilities \citep[see, e.g.,][ or
\citet{bollerslevtodorovxu15}]{Bollerslev.2013} or, more generally, an
increase in the marginal investors' risk aversion \citep{Bekaert.2009}.
In response, the marginal investor may require higher compensation to
hold the asset \citep[\citet{Campbell.1999},
\citet{Martin.2017}]{campbellgrossmanwang93}. A potential equilibrium
induces a flight to safety by disposing of risky assets
\citep[\citet{beberbrandtkavajecz09}, \citet{Adrian.2019},
\citet{Baele.2019}]{longstaff04}.

A \emph{VIX} shock may also indicate an \emph{ERV} shock. If there is
news that affects a firm's cash flow streams, trading generates
volatility. Liquidity providers as counterparties to informed trading
are exposed to adverse selection. As a response, a \emph{VIX} shock due
to news manifesting in higher expected realized variance may render
markets illiquid \citep[\citet{drechsler.2022}]{nagel12}.

In this paper, we argue that \emph{VRP} and \emph{ERV} can be subject to
different shocks that govern \emph{VIX} dynamics. Market responses to a
\emph{VIX} shock are therefore contingent on the nature of the shock.
Indeed, there is strong evidence that \emph{VRP} and \emph{ERV} impulses
are not perfectly
correlated.\footnote{For example, a large literature shows that central bank communication mainly impacts risk premia rather than expected realized variance \citep{Bernanke.2005, Drechsler.2018}. Such a shift in preferences can generate a time-varying wedge between \emph{ERV} and \emph{VRP} \citep{bekaerthoerova2021}.}

To understand market dynamics, we follow \cite{bekaerthoerova14} and
decompose high-frequency \emph{VIX} changes into two canonical
components: changes in the variance risk premium and changes in expected
realized variance. We analyze market responses to \emph{ERV} and
\emph{VRP} shocks based on an exhaustive 2007-2021 sample of \emph{all}
NASDAQ trading messages for two exchange-traded funds (ETFs): SPY for
the S\&P 500 equity index and TLT for government bonds. Employing a a
vector autoregressive (VAR) model, we quantify the impulse responses due
to shocks in either \emph{ERV} or \emph{VRP} on the (net) trading
volume, midquote returns, bid-ask spreads, order book depth, and the
\cite{Amihud.2002} illiquidity ratio.

We show that a shift in investors' valuation of crash-risk states (the
\emph{VRP} channel) triggers fundamentally different market responses
than a shift in expected realized variances (\emph{ERV} channel): The
most salient finding is a flight to safety in response to a \emph{VRP}
shock at essentially unchanged liquidity. Prices drop for equities and
increase for government bonds. These price responses revert partially in
the subsequent hour despite the continued selling of equities and buying
of bonds. In contrast, in response to an \emph{ERV} shock, we observe a
decline in liquidity and \emph{net buying} of the risky asset. The
fragility in liquidity is reflected by a sizeable jump in the
\cite{Amihud.2002} illiquidity ratio.

Our empirical findings are robust across several dimensions: The results
are not driven by estimation error in \emph{ERV}, which potentially is
absorbed by \emph{VRP}: The responses remain consistent even if we
impose perfect foresight such that the investor knows the future
realized variance already ex-ante the shock.\\
In fact, the responses increase in magnitude during periods of general
market turmoil, that is, when we restrict our sample to the global
financial crisis and the 2020 COVID-19 outbreak. Our results are also
consistent with trading patterns of institutional trading:\\
For 2009 through 2013, we use standard Abel Noser data on institutional
order flow to redo the impulse-response analysis. We find statistically
significant institutional selling of equities on \emph{VRP} shocks and
significant institutional \emph{buying} on \emph{ERV} shocks.

Finally, the net buying of the S\&P\textasciitilde500 in response to an
\emph{ERV} shock is not an indication of a flight-to-US due to a
simultaneous disposal of global equity holdings: The responses are
consistent for other risky assets such as a well-diversified global
equity portfolio as well as ETFs tracking baskets of corporate bonds.\\
The nature of the shock to \emph{VIX} seems to matter empirically since
the response to \emph{VRP} shocks differs from the response to
\emph{ERV} shocks.

Intuition for these empirical patterns are gained in the context of an
information asymmetry model of liquidity
\citep[e.g., ][]{grossmanstiglitz80}: Investors who suddenly become more
risk averse become active liquidity demanders and re-allocate risk to
those who stay equally risk averse. These resulting liquidity demanders
actively sell equities and buy government bonds. Equity prices drop
since the average investor becomes more risk-averse. This type of shock
does not affect liquidity because information asymmetry is unaffected.

The intuition for the risk (\emph{ERV}) channel is as follows. The net
buying of equities by liquidity demanders might be due to the uninformed
market participants experiencing additional risk for the simple reason
that they are \emph{uninformed}. They understand that there is news, and
they realize that others have access to it. They, therefore, prefer to
reduce their equity holdings. The (informed) liquidity demanders observe
the news and, therefore, bear less posterior risk. They sell on bad
news, they buy on good news, but asymmetrically so, because,
\emph{on average}, they must be buyers in equilibrium. The reason is
that the risk shock increases the wedge between the risk that the
informed experience relative to the risk that the uninformed experience.
In this case, however, liquidity \emph{is} affected. It worsens due to
an increased information asymmetry between liquidity demanders and
liquidity suppliers.

In summary, we find evidence for \emph{VIX} shocks causing flight to
safety and market fragility, but due to separate channels. Shocks to the
variance risk premium (\emph{VRP}) trigger active re-allocation from
equities to government bonds, without impairing market liquidity. Shocks
to the expected realized variance (\emph{ERV}), do not trigger active
re-allocation but impair market liquidity, at least contemporaneously. A
-- on the first sight -- surprising (but rationalized) finding is the
\emph{active buying} of equities that accompanies this market fragility.

Our results have (at least) two relevant implications: First, after the
global financial crisis, central banks are concerned about market
liquidity as part of their focus on liquidity risk management
\citep{bis.2010}.\footnote{The SEC worries about the ramifications of liquidity spirals as evidenced by their 2016 rule that requires each registered open-end management investment company, including open-end exchange-traded funds (ETFs), to establish a liquidity risk management program \citep{sec16}. European regulators share the worry but stopped short of imposing regulation \citep{esma19}.}
Our findings show that liquidity management is particularly sensitive to
shocks in \emph{ERV}.

Second, a growing literature focuses on the effect of central bank
communication on financial markets. Existing work indicates that central
bank announcements affect mainly \emph{VRP} instead of \emph{ERV}, i.e.,
central bank communication changes investors' perception of crash risk
probabilities rather than the expected realized variance. Our results
indicate that such a focus tends to keep markets stable in the sense
that liquidity is not directly affected.

Our findings add to a rapidly growing asset-pricing literature on the
role of volatility. \cite{Ait.2021} motivate their model by quoting a
New York Fed President who in 2017 wondered: ``You would think if
uncertainty was high, you'd have a bit more volatility.'\,' In their
paper, they rationalize asset-price dynamics by introducing two
disconnected stochastic processes: One that drives (realized) volatility
and another that drives risk (aversion). Models in the same spirit are
\cite{Liu.2004}, \cite{Drechsler.2013}, and \cite{Brenner.2018}. Our
results complements this literature by identifying the different natures
of the two fundamental channels of VIX shocks.

The manuscript is organized as follows.
Section\textasciitilde{}\ref{sec:example} develops intuition for what
drives changes in \emph{VIX}, in \emph{ERV}, and in the wedge between
them, which corresponds to \emph{VRP}. We provide a simple framework
starting from the generic stochastic pricing kernel representation and
show that crash risk probabilities, governed by more negative return
skewness and an increase in the risk aversion lead to an increase of
\emph{VRP}, while an increase of \emph{ERV} leads to ambiguous results
for the \emph{VRP}. Section\textasciitilde{}\ref{sec:data} presents the
intraday \emph{VIX} decomposition and the order book data.
Section\textasciitilde{}\ref{sec:methodology} discusses the estimation
of impulse-response functions, along with appropriate confidence
intervals. Section\textasciitilde{}\ref{sec:results} presents the main
empirical findings: Market responses to \emph{VIX} (component) impulses.
Section\textasciitilde{}\ref{sec:conclusion} concludes.


\renewcommand\refname{The wedge between expected realized variance under
\(\mathbbm{P}\) and \(\mathbbm{Q}\) \label{sec:example}}
\bibliography{bibliography.bib}



\end{document}
