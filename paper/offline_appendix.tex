\documentclass[12pt]{article}
\usepackage{setspace}
\onehalfspacing
\usepackage[a4paper, left=1in, right=1in, top=1.5in, bottom=1.5in]{geometry}

% --- More readable font on screens
\usepackage{amsmath, amsthm, amssymb}
\usepackage{xcolor}
\setlength {\marginparwidth }{43mm} % Needed to avoid warnings from todonotes
%\usepackage[color=black!10, linecolor=black]{todonotes}
\usepackage{amsthm}
\usepackage{txfonts}
\usepackage{pdflscape}

\usepackage{rotating}
\usepackage{multirow}

\usepackage[colorlinks = true, 
urlcolor = black, 
linkcolor = black, 
citecolor = black]{hyperref}

\usepackage{bm}
\usepackage{bbm}
\usepackage{float}
\usepackage{pdflscape}

\usepackage{graphicx}
\usepackage{booktabs}
\usepackage{tabularx}
\usepackage{ltxtable}
\usepackage{placeins}
\usepackage{changepage}
\usepackage{pdfpages}
\usepackage{dsfont}

\usepackage[labelfont={bf}, font = {scriptsize}, textfont={scriptsize}, justification=raggedright]{caption}

\newtheorem{proposition}{Proposition}
\newtheorem{lemma}{Lemma}
\newtheorem{corollary}{Corollary}

\usepackage[round]{natbib}
\bibliographystyle{chicago}

\usepackage{setspace}

% --- Formatting a typical pagehttps://www.overleaf.com/project/5e95a73222bf470001c93026
% https://tex.stackexchange.com/questions/59626/nicely-force-66-characters-per-line
%\usepackage{geometry} % Follow style guidelines on formatting, easier to read on paper and tablet.  It now obeys The Elements of Typographic Style, see https://www.amazon.com/dp/0881792063.
\newcommand*\GetTextWidth[3][\normalfont]{{#1%
    \settowidth{#2}{abcdefghijklmnopqrstuvwxyz}%
    \setlength{#2}{0.03193#2}%
    \addtolength{#2}{0.44961pt}%
    \setlength{#2}{#3#2}%
    \global#2=#2}}
\newlength\bringhurstwdt
\GetTextWidth{\bringhurstwdt}{66}
\geometry{textwidth=\bringhurstwdt, heightrounded}

\newcolumntype{Y}{>{\centering\arraybackslash}X}

\newcommand\invisiblesection[1]{%
	\refstepcounter{section}%
	\addcontentsline{toc}{section}{\protect\numberline{\thesection}#1}%
	\sectionmark{#1}}

\newcommand{\ourtitle}{Market Response to A VIX Impulse}

\begin{document}
	\hypersetup{pageanchor=false}

\hypersetup{pageanchor=true}
\setcounter{page}{1}	
\thispagestyle{empty}

\begin{center}
	\large Offline Appendix to \\
	\vspace{1.5cm}
	\LARGE{\ourtitle}
\end{center}

\vspace{0.75cm}

\begin{center}
	\large
	\begin{tabularx}{1\textwidth}{YYY}
  	Nikolaus Hautsch  & Albert J. Menkveld & Stefan Voigt 
	\end{tabularx}
\end{center}

\vspace{0.75cm}

\normalsize

\noindent \textbf{Internet Appendix \ref{ia:standardized_irf}:} \:\:\:\:\: Standardized Impulse Response functions

\vspace{1.50 cm}

\noindent \textbf{Internet Appendix \ref{ia:asymmetry}:} \:\:\:\:\: Asymmetric Responses

\vspace{1.50 cm}

\noindent \textbf{Internet Appendix \ref{ia:subsample_analysis}:} \:\:\:\:\: Rolling estimation windows

\vspace{1.50 cm}

\noindent \textbf{Internet Appendix \ref{ia:vrp}:} \:\:\:\:\: Variance Risk Premium

\newpage
\setcounter{page}{1}
\renewcommand*{\thepage}{IA \arabic{page}}

\section{IV decomposition summary statistics}

\begin{landscape}
\begin{table}
\centering
\resizebox{\linewidth}{!}{
\begin{tabular}{llrrrrrrrrrrrrrrrr}
\toprule
  &   & 2007 & 2008 & 2009 & 2010 & 2011 & 2012 & 2013 & 2014 & 2015 & 2016 & 2017 & 2018 & 2019 & 2020 & 2021 & Total\\
\midrule
 & Mean & 15.309 & 55.071 & 56.769 & 17.968 & 22.689 & 13.653 & 9.086 & 7.255 & 9.028 & 12.895 & 5.240 & 11.228 & 14.477 & 38.202 & 34.014 & 21.047\\

 & Median & 15.150 & 31.022 & 50.160 & 18.217 & 17.009 & 12.952 & 8.952 & 7.282 & 8.925 & 12.434 & 5.108 & 8.934 & 13.315 & 32.281 & 29.891 & 12.832\\

 & SD & 5.880 & 56.039 & 37.600 & 5.939 & 11.963 & 4.595 & 1.321 & 1.353 & 2.398 & 5.074 & 1.127 & 7.088 & 5.059 & 28.011 & 12.648 & 26.308\\

\multirow{-4}{*}{\raggedright\arraybackslash ERV} & IQR & 10.131 & 20.530 & 54.202 & 8.046 & 18.811 & 7.058 & 1.713 & 1.761 & 3.046 & 7.228 & 1.849 & 10.778 & 6.891 & 29.313 & 14.752 & 14.041\\
\cmidrule{1-18}
 & Mean & -0.011 & -0.003 & -0.060 & -0.006 & -0.005 & -0.005 & -0.004 & -0.003 & -0.003 & -0.007 & -0.002 & 0.000 & -0.009 & -0.014 & 0.007 & -0.009\\

 & Median & -0.018 & -0.052 & -0.057 & -0.014 & -0.013 & -0.011 & -0.008 & -0.005 & -0.010 & -0.016 & -0.004 & -0.008 & -0.011 & -0.023 & -0.036 & -0.010\\

 & SD & 0.204 & 0.995 & 0.746 & 0.111 & 0.141 & 0.078 & 0.045 & 0.029 & 0.056 & 0.092 & 0.020 & 0.090 & 0.090 & 0.337 & 0.361 & 0.361\\

\multirow{-4}{*}{\raggedright\arraybackslash ERV (change)} & IQR & 0.070 & 0.223 & 0.266 & 0.052 & 0.061 & 0.034 & 0.018 & 0.013 & 0.027 & 0.040 & 0.009 & 0.033 & 0.041 & 0.121 & 0.101 & 0.043\\
\cmidrule{1-18}
 & Mean & 41.610 & 111.385 & 90.362 & 44.664 & 54.337 & 27.266 & 17.170 & 17.299 & 24.979 & 22.501 & 10.427 & 25.046 & 20.672 & 85.240 & 43.285 & 42.364\\

 & Median & 41.070 & 52.334 & 67.403 & 39.749 & 35.673 & 25.901 & 15.847 & 15.732 & 20.228 & 17.569 & 9.792 & 19.815 & 18.650 & 60.615 & 40.959 & 25.638\\

 & SD & 14.204 & 119.563 & 53.021 & 22.216 & 37.332 & 8.043 & 4.253 & 7.040 & 14.475 & 12.341 & 2.615 & 16.461 & 7.493 & 84.964 & 12.243 & 54.088\\

\multirow{-4}{*}{\raggedright\arraybackslash IV} & IQR & 21.008 & 94.775 & 84.422 & 24.212 & 58.334 & 9.878 & 4.636 & 6.634 & 12.076 & 10.943 & 3.100 & 19.238 & 9.459 & 48.424 & 10.264 & 29.886\\
\cmidrule{1-18}
 & Mean & -0.002 & -0.013 & -0.009 & -0.002 & 0.002 & -0.002 & -0.001 & 0.002 & -0.005 & -0.004 & -0.001 & 0.006 & -0.007 & -0.010 & -0.004 & -0.003\\

 & Median & 0.000 & 0.000 & 0.000 & 0.000 & 0.000 & 0.000 & 0.000 & 0.000 & 0.000 & 0.000 & 0.000 & 0.000 & 0.000 & 0.000 & -0.029 & 0.000\\

 & SD & 0.626 & 2.454 & 0.923 & 0.757 & 0.923 & 0.296 & 0.162 & 0.348 & 0.688 & 0.356 & 0.151 & 1.085 & 0.329 & 2.733 & 1.021 & 1.153\\

\multirow{-4}{*}{\raggedright\arraybackslash IV (change)} & IQR & 0.457 & 0.572 & 0.469 & 0.252 & 0.310 & 0.139 & 0.090 & 0.123 & 0.260 & 0.214 & 0.106 & 0.280 & 0.208 & 0.691 & 0.483 & 0.239\\
\cmidrule{1-18}
 & Mean & 26.301 & 56.314 & 33.593 & 26.697 & 31.648 & 13.613 & 8.084 & 10.044 & 15.951 & 9.606 & 5.187 & 13.818 & 6.195 & 47.037 & 9.271 & 21.317\\

 & Median & 23.622 & 20.694 & 26.704 & 20.458 & 17.089 & 12.867 & 6.995 & 8.404 & 10.558 & 6.415 & 4.566 & 8.372 & 5.633 & 24.247 & 8.897 & 11.358\\

 & SD & 11.537 & 80.447 & 25.876 & 20.592 & 29.227 & 5.261 & 3.725 & 6.487 & 14.886 & 9.086 & 2.188 & 13.594 & 3.973 & 83.999 & 8.775 & 37.535\\

\multirow{-4}{*}{\raggedright\arraybackslash VRP} & IQR & 14.584 & 45.716 & 24.821 & 18.749 & 35.091 & 6.202 & 3.770 & 5.960 & 10.617 & 6.755 & 1.995 & 12.039 & 4.782 & 29.318 & 11.340 & 15.528\\
\cmidrule{1-18}
 & Mean & 0.009 & -0.010 & 0.051 & 0.004 & 0.007 & 0.003 & 0.003 & 0.004 & -0.002 & 0.004 & 0.002 & 0.007 & 0.002 & 0.004 & -0.012 & 0.006\\

 & Median & 0.034 & 0.076 & 0.087 & 0.013 & 0.022 & 0.015 & 0.009 & 0.007 & 0.014 & 0.021 & 0.003 & 0.011 & 0.012 & 0.042 & 0.045 & 0.016\\

 & SD & 0.660 & 2.682 & 1.190 & 0.762 & 0.935 & 0.307 & 0.166 & 0.348 & 0.690 & 0.367 & 0.151 & 1.085 & 0.336 & 2.741 & 1.062 & 1.211\\

\multirow{-4}{*}{\raggedright\arraybackslash VRP (change)} & IQR & 0.466 & 0.625 & 0.564 & 0.271 & 0.325 & 0.147 & 0.095 & 0.125 & 0.265 & 0.218 & 0.106 & 0.281 & 0.221 & 0.715 & 0.524 & 0.251\\
\bottomrule
\end{tabular}}
\end{table}\end{landscape}



\section{Standardized Impulse Response functions}\label{ia:standardized_irf}

Figure~\ref{fig:appendix_vix_irf_vix} reports the response of (cumulative) \textit{IV} changes to a \textit{IV} change increase. \textit{IV} shocks are persistent, in the sense that a one standard deviation \textit{IV} shift triggers a further 8\% increase during the next 20 minutes and then slowly decays again.

Figure~\ref{fig:appendix_full_irf_vix} illustrates the standardized responses to a one standard deviation \textit{IV} change increase. For that purpose, we demean and standardize every time series such that it exhibits unit standard deviation before we compute the impulse response functions. The units of the estimated coefficients can thus be interpreted in terms of standard deviations.


\begin{figure}
    \centering
    \includegraphics[width = 0.88\linewidth]{offline_figures/iv_to_iv_responses.jpeg}
    \caption{(Standardized) estimated cumulative impulse response functions based on the entire sample (July 1st, 2007 until April 7th, 2021) for \textit{IV changes}. We compute the responses after a positive one standard deviation \textit{IV change} shock. The $x$-axis denotes the \textit{horizon} in minutes, ranging from instantaneous (0-minute) up to 60-minute responses. The error bars illustrate 95\% confidence intervals.}
    \label{fig:appendix_vix_irf_vix}
 \end{figure}
 
  \begin{figure}
    \centering
    \includegraphics[width = 0.88\linewidth]{offline_figures/irf_full_iv_standardized.jpeg}\caption{Standardized estimated impulse response functions based on the entire sample (July 1st, 2007 until April 7th, 2021). We shock \textit{IV changes} by one positive standard deviation. \textit{Returns} and \textit{initiator net volume} responses denote \textit{cumulative} responses, whereas we report impulse responses for \textit{trading volume}, \textit{depth}, \textit{bid-ask spread} and \textit{ILLIQ}. We report all responses in terms of standard deviations. \textit{Horizon} denotes the instantaneous (0 minute), 20 minute, 40 minute and 60 minute responses. The error bars illustrate 95\% confidence intervals.}
    \label{fig:appendix_full_irf_vix}
 \end{figure}

\FloatBarrier

\section{Asymmetric responses}\label{ia:asymmetry}
 
 To address potentially asymmetric responses to buy-side or sell-side initiator volume in S\&P 500 trading, we decompose \textit{IV changes} into a positive and negative component and analyze the responses in a system where we shock only one side of initiator net volume. More specifically, we construct two time series for \textit{IV changes}:
 \begin{align}
     \textit{Pos IV Change}_{t, \tau} =      \textit{IV Change}_{t, \tau} \times \mathds{1}_{\textit{IV Change}_{t, \tau} > 0}\\
    \textit{Neg IV Change}_{t, \tau} =      \textit{IV Change}_{t, \tau} \times \mathds{1}_{\textit{IV Change}_{t, \tau} < 0}.
 \end{align}
The definition above ensures that 
\begin{equation}
    \textit{IV Change}_{t, \tau} = \textit{Pos IV Change}_{t, \tau} + \textit{Neg IV Change}_{t, \tau}.
\end{equation}
We compute the impulse response functions for the entire system with $N=24$ order book variables and shock positive $\textit{IV Change}_{t, \tau}$ and negative $\textit{IV Change}_{t, \tau}$ separately.
To obtain comparable results, we impose a negative shock on $\textit{Neg IV Change}_{t, \tau}$ (which resembles a decrease in IV) and a positive shock on $\textit{Pos IV Change}_{t, \tau}$. 
Figure~\ref{fig:irf_compare_asymmetry_vix} illustrates the results for the separate estimations. The figure illustrates that liquidity (depth and bid-ask spreads) reacts especially strong after an \textit{IV} increase. In fact, after an \textit{IV} decrease, bid-ask spreads for the S\&P 500 and Government Bonds even decrease after 60 minutes whereas they remain elevated after an \textit{IV} increase. Depth is lower for all assets, whereas the instantaneous effect for the S\&P 500 is three times as strong after a positive \textit{IV} shock relative to a negative shock.  

 \begin{figure}
    \centering
    \includegraphics[width = 0.88\linewidth]{offline_figures/irf_asymmetry_iv_raw.jpeg}
    \caption{Estimated impulse response functions based on the entire sample (July 1st, 2007 until April 7th, 2021). We shock \textit{IV} changes \textit{positive} by one positive standard deviation and \textit{negative} \textit{initiator net volume} by one negative standard deviation. \textit{Returns} and \textit{initiator net volume} responses denote \textit{cumulative} responses, whereas we report impulse responses for \textit{trading volume}, \textit{depth}, \textit{bid-ask spread} and the Amihud \textit{ILLIQ} measure.  We report \textit{Return} and \textit{bid-ask spread} responses in basis points and denote all remaining variables in million USD. \textit{Horizon} denotes the instantaneous (0 minute), 20 minute, 40 minute and 60 minute responses. The error bars illustrate 95\% confidence intervals.}
    \label{fig:irf_compare_asymmetry_vix}
 \end{figure}
\FloatBarrier
\section{Rolling estimation windows}\label{ia:subsample_analysis}

Figure~\ref{fig:ts_irf_vix} illustrates rolling window estimates of the impulse response function coefficients. To compute rolling-window impulse response, we always impose a positive one standard deviation shock on \textit{IV, ERV, and VRP changes} (based on the preceding 12 months). We estimate the impulse response functions for each day in our sample (in total 3460 trading days), with lag-augmented local projections. We estimate the coefficients based on a rolling window of length 12 months. The figure establishes a number of stylized facts that seem to characterize the relationship between risk, liquidity, and portfolio choice on a high-frequency basis: Tendencies for flight to safety as a response to a sudden \textit{IV} shock are present throughout the entire 14-year sample period. We find consistent negative responses for initiator net volume for S\&P 500 and the positive responses for government bonds. 
 \begin{figure}
    \centering
    \includegraphics[width = \linewidth]{offline_figures/irf_rolling_window.jpeg}
    \caption{Time series of impulse response functions based on a rolling estimation window of 12 months length. \textit{IV} are the standardized implied variance changes, \textit{ERV} is the standardized change in expected realized volatility and \textit{VRP} is the standardized change in the variance risk premium, measured as $\widetilde{VRP}_{t, \tau} = IV_{t, \tau} - \widetilde{E}_{t, \tau}\left(RV_{t, \tau}^{(22)}\right)$. \textit{Returns} and \textit{initiator net volume} responses denote \textit{cumulative} responses, whereas for \textit{trading volume}, \textit{depth}, \textit{bid-ask spread} and \textit{ILLIQ} we report impulse responses. We report all instantaneous (0-minute) responses.}
    \label{fig:ts_irf_vix}
 \end{figure}

\FloatBarrier


\section{Variance Risk Premium}\label{ia:vrp}

In order to analyze the relevance of other potential forecasting candidates based on the limit order book data we extend the predictive regression in Equation~\eqref{eq:rv_regression} and estimate LASSO-regression coefficients of the following extended regression: 
{\scriptsize \begin{equation*}
     \log\left(RV_{t, \tau}^{(22)}\right)= c_\tau + \beta_\tau \log\left(RV_{t-22, 0}^{(21)} + \widetilde{RV}_{t, \tau}\right) + \gamma_\tau \log\left(RV_{t-5, 0}^{(4)} + \widetilde{RV}_{t, \tau}\right) + \delta_\tau \log\left(\widetilde{RV}_{t, \tau}\right) + \omega_\tau' X_{t, \tau} + \varepsilon_{t, \tau}
\end{equation*}}
where $X_{t,\tau}$ is a $(60 \times 1)$ vector that contains \textit{i)} the 24 limit order book observations at day $t$ and timestamp $\tau$ from Section~\ref{sec:data}, \textit{ii)} the levels of the \textit{VIX} and \textit{SPX} indices, the \emph{SPX} return together with $RV_{t-22, 0}^{(21)}$, $RV_{t-5, 0}^{(4)}$, $\widetilde{RV}_{t, \tau}$, and \textit{iii)} all squared terms. LASSO (the least absolute shrinkage and selection operator) performs variable selection in the sense that the objective is not simply to choose the vector of parameters $\theta^\tau = \left(c_\tau, \beta_\tau, \ldots, \omega_\tau\right)$ that minimizes the sum of squared errors, but instead imposes a penalty on the sum of the absolute values of the regression parameters $\lambda\left\|\left(c_\tau', \beta_\tau', \gamma_\tau', \delta_\tau', \omega_\tau'\right)\right\|_{L_1}$. We select the hyperparameter $\lambda$ with 10-fold cross-validation. Table \ref{tab:lasso_results} illustrates the frequency with which the individual parameters have been selected (based on the 78 individual regressions). The table illustrates that most models select the realized variance of the past week, the realized intraday variance and the monthly realized variance which constitute our benchmark. Incorporating trading volume (\textit{S\&P 500}) seems to improve predictive accuracy in some cases as well. All our results remain qualitatively similar, however, if we include the selected limit order book information into our predictive framework. 

\begin{table}
\centering
\begin{tabular}{lr}
\toprule
Variable & Selection frequency\\
\midrule
$\log\left(RV_{t-5, 0}^{(4)}\right)$ & 97.73\\
$\log\left(\widetilde{RV}_{t,\tau}\right)$ & 67.42\\
$\log\left(RV_{t-22, 0}^{(21)}\right)$ & 62.88\\
Trading volume (S\&P 500) & 55.30\\
VIX & 53.79\\
Spread (Junk Bonds) & 51.52\\
Depth (S\&P 500) & 50.00\\
Depth (Junk Bonds) & 49.24\\
SPX & 46.97\\
Spread (Corporate Bonds) & 43.94\\
\bottomrule
\end{tabular}
\caption{Lasso selection frequencies. The table illustrates the frequency with which the most relevant variables have been chosen in the extended regression, including contemporaneous limit order book information.}
\label{tab:lasso_results}
\end{table}
\end{document}